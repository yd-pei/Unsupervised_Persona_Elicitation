\documentclass{article}

% Recommended packages
\usepackage[utf8]{inputenc} % allow utf-8 input
\usepackage[T1]{fontenc}    % use 8-bit T1 fonts
\usepackage{hyperref}       % hyperlinks
\usepackage{url}            % simple URL typesetting
\usepackage{booktabs}       % professional-quality tables
\usepackage{amsfonts}       % blackboard math symbols
\usepackage{nicefrac}       % compact symbols for 1/2, etc.
\usepackage{microtype}      % microtypography
\usepackage{graphicx}       % for figures
\usepackage{amsmath}        % for math formulas
\usepackage{algorithm}      % for algorithms
\usepackage{algorithmic}    % for algorithms

\title{Unsupervised Persona Elicitation}

\author{
  Yiding Pei \\
}

\date{\today}

\begin{document}

\maketitle

\section{Results}


\section{Experimental Setup}
\label{sec:setup}
Describe how the experiments were conducted.

\subsection{Datasets}
Describe the testbeds used (e.g., Global Opinions).
\begin{itemize}
    \item \textbf{Global Opinions}: Description of the dataset, size, and characteristics.
    \item \textbf{Other Datasets}: If any.
\end{itemize}

\subsection{Models}
List the LLMs used in the experiments (e.g., Meta-Llama).

\subsection{Baselines}
Describe the baseline methods compared against (e.g., Zero-shot, Few-shot with random labels, etc.).

\subsection{Evaluation Metrics}
Define the metrics used to evaluate performance (e.g., Accuracy, Consistency, Agreement).

\begin{table}[h]
    \centering
    \caption{Main performance comparison on Global Opinions.}
    \begin{tabular}{lccc}
        \toprule
        Method & Accuracy & Consistency & Metric 3 \\
        \midrule
        Baseline 1 & 0.00 & 0.00 & 0.00 \\
        Baseline 2 & 0.00 & 0.00 & 0.00 \\
        \textbf{Ours} & \textbf{0.00} & \textbf{0.00} & \textbf{0.00} \\
        \bottomrule
    \end{tabular}
    \label{tab:main_results}
\end{table}


\section{Discussion}
\label{sec:discussion}
Interpret the results. Why does the unsupervised method work? What are the limitations?
\begin{itemize}
    \item \textbf{Analysis of Learned Personas}: Qualitative examples of elicited personas.
    \item \textbf{Limitations}: Where does the method fail?
\end{itemize}

\section{Critique}
\label{sec:conclusion}
Here we discuss some limitations of the ICM method.

\subsection{TruthfulQA dataset}

\subsection{Robustness against adversarial prompts}

\subsection{Ablation experiment}

\bibliographystyle{plain}
\bibliography{references} % Assumes a references.bib file exists

\end{document}
